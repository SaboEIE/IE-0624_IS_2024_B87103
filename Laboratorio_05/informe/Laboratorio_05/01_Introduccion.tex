\newpage
\section{INTRODUCCIÓN}
 Este trabajo se centra en el desarrollo de un clasificador y detector de mascarillas faciales utilizando el kit Arduino Nano 33 BLE con la cámara OV7675, con el objetivo de diseñar y entrenar un modelo que pueda detectar si una persona está usando una mascarilla, ofreciendo una solución eficiente y de bajo costo para reforzar las medidas de salud pública. El desarrollo del proyecto comenzó con la construcción de un conjunto de datos, capturando 20 imágenes de personas con mascarillas, 20 imágenes de personas sin mascarillas y 20 imágenes sin ninguna persona frente a la cámara, variando parámetros como el ángulo de la cabeza y el color de la mascarilla. Posteriormente, se diseñó el impulso en Edge Impulse, utilizando un bloque de preprocesamiento de imagen configurado para trabajar con imágenes en RGB o en escala de grises, generando vectores de características y empleando un bloque de modelo de transferencia configurado con parámetros específicos como el número de ciclos de entrenamiento, la tasa de aprendizaje y el aumento de datos. El modelo se entrenó considerando las características del microcontrolador, como la memoria ROM y RAM, y se validó exhaustivamente con pruebas para afinar su precisión y capacidad de generalización. La implementación en el microcontrolador incluyó una lógica de respuesta utilizando un LED RGB: azul cuando no se detectaba una cara, rojo para una cara sin mascarilla y verde para una cara con mascarilla, proporcionando una respuesta visual clara y efectiva. Los resultados mostraron una precisión del 92.3\% en el entrenamiento y del 80.77\% en la validación, indicando una alta eficacia en la identificación de las clases definidas. Aunque los resultados fueron satisfactorios, se recomendó explorar técnicas adicionales para mejorar el rendimiento del modelo, como diferentes arquitecturas de redes neuronales y el aumento del conjunto de datos, así como implementar medidas para mitigar desafíos como el ruido en los datos o las variaciones ambientales, para asegurar un rendimiento consistente en condiciones del mundo real. Este trabajo demuestra que es posible desarrollar un clasificador eficiente de mascarillas faciales utilizando herramientas accesibles y de bajo costo, ofreciendo una solución potencialmente valiosa para entornos donde la salud pública es una prioridad.


Para consultar el trabajo realizado, puede consultarse el siguiente repositorio de trabajo, en la rama llamada ``Lab5'': \url{https://github.com/SaboEIE/IE-0624_IS_2024_B87103.git}. El código principal en C (archivo .ino), puede consultarse en la ruta \path{Laboratorio_05/src}. 
 