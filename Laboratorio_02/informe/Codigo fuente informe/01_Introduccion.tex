\newpage
\section{INTRODUCCIÓN}
Dentro del vasto terreno de la electrónica digital y la computación embebida, los microcontroladores destacan como elementos cruciales. Estos dispositivos, al reunir en un solo chip un procesador, memoria, periféricos de entrada/salida y otras funcionalidades, se posiciona como la columna vertebral de una amplia gama de aplicaciones tecnológicas. Su capacidad para gestionar y coordinar diversas tareas en tiempo real los convierte en componentes esenciales en la actual era digital, donde la demanda de sistemas compactos, rápidos y energéticamente eficientes es cada vez mayor. El objetivo de este laboratorio es construir un simulador de un cruce peatonal de unidireccional, utilizando como base el microcontrolador ATtiny4313, así como el lenguaje de programación C y algunos elementos pasivos y activos que complementan el diseño.  

A lo largo de este informe se documenta la forma en la que fueron puestos en práctica una serie de conceptos básicos de GPIO, \textit{timers} y máquinas de estado finito. Lo anterior se logró utilizando el lenguaje de programación C para la construcción de la lógica computacional de fondo, el simulador SimulIDE para diseñar el circuito y realizar las pruebas de funcionamiento, además de los conocimiento adquiridos a lo largo de la carrera sobre circuitos lineales y electrónica. Asimismo, se detalla el proceso de cálculo de las magnitudes físicas de los componentes utilizados, así como la lógica de programación utilizada en las funciones construidas para llegar al resultado final. Por último, se muestran los resultados obtenidos en los que puede verse un simulador de cruce peatonal completamente funcional y apegado a las especificaciones solicitadas en el enunciado. Para consultar el trabajo realizado, puede consultarse el siguiente repositorio de trabajo: \url{https://gitlab.com/arbre29/laboratorio_de_microcontroladores.git}. El código en C, archivo de simulación .simu, el archivo READ.md y el Makefile puede consultarse en la ruta \path{Laboratorio_02/src}. 