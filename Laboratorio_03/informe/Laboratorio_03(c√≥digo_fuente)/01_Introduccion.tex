\newpage
\section{INTRODUCCIÓN}
 Este trabajo aborda el desarrollo de una incubadora de huevos automatizada. Utilizando la plataforma Arduino UNO, se diseña con el objetivo principal de mantener la temperatura óptima para la incubación de huevos dentro del rango [30, 42]°C. Para alcanzar este propósito, se implementó un control PID que regula la temperatura, siendo configurado mediante prueba/error para garantizar un desempeño eficiente. El sistema incluye una pantalla LCD PCD8544 para visualizar la temperatura actual, el valor de control del PID y la temperatura medida. Además, se han incorporado LEDs de alarma para indicar situaciones fuera del rango óptimo de temperatura: azul para temperaturas inferiores a 30°C, rojo para temperaturas superiores a 42°C y verde para temperaturas dentro del rango deseado.
 Además se ha integrado un sistema que permite el registro de datos de temperatura en un archivo de texto plano. Este registro se realiza mediante la comunicación serial con un programa Python que captura y almacena los datos en formato CSV, además de proporcionar visualización gráfica de la temperatura de referencia, la salida del PID y la salida del sistema de control. El análisis de las gráficas obtenidas muestra un comportamiento satisfactorio del sistema diseñado. Los resultados validan la efectividad del diseño propuesto para la incubadora automatizada, demostrando su capacidad para mantener condiciones estables de temperatura dentro del rango requerido para la incubación de huevos.
 Para consultar el trabajo realizado, puede consultarse el siguiente repositorio de trabajo, en la rama llamada ``Lab3'': \url{https://github.com/SaboEIE/IE-0624_IS_2024_B87103}. El código en C, archivo de simulación .simu, el archivo READ.md y el Makefile puede consultarse en la ruta \path{Laboratorio_03/src}. 
 