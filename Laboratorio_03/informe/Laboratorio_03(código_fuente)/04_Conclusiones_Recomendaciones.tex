\newpage
\section{CONCLUSIONES Y RECOMENDACIONES}
A lo largo del desarrollo del informe se pusieron a prueba una serie de conceptos y prácticas importantes para el manejo básico del Arduino UNO, así como sus GPIO, convertidores analógico-digitales, comunicación serial y formas de programación. A continuación algunas conclusiones y recomendaciones recolectadas a lo largo de trabajo realizado: 

\begin{itemize}
    \item Se confirma que el sistema responde correctamente a diferentes rangos de temperatura, como se evidencia en la activación de los LEDs correspondientes dependiendo de si la temperatura está por debajo, dentro o por encima del rango óptimo. Esto indica una adecuada detección y respuesta del sistema a las condiciones térmicas requeridas.

    \item  Se observa que la desactivación de la pantalla y la comunicación serial ocurren como se espera cuando los respectivos \textit{switches} no son activados. Esto sugiere un buen funcionamiento de las funciones adicionales de control y comunicación, lo cual es esencial para el funcionamiento global del sistema.

    \item Los resultados de aumento y disminución de temperatura muestran una respuesta coherente del sistema, donde la señal de control se ajusta apropiadamente para mantener la temperatura de salida cerca de la temperatura de referencia. Esto indica una efectiva regulación térmica y una capacidad de respuesta dinámica del sistema.

    \item Es recomendable realizar pruebas adicionales en condiciones extremas o inesperadas para evaluar la robustez del sistema. Esto ayudaría a identificar posibles vulnerabilidades o escenarios no contemplados durante el diseño inicial.

    \item Dado el buen funcionamiento general del sistema, sería beneficioso agregar alarmas o notificaciones para alertar a los usuarios en caso de que la temperatura se salga del rango óptimo durante un tiempo prudencial. Esto es especialmente crítico en el caso de una incubadora, donde desviaciones de temperatura podrían dañar los huevos en desarrollo.
\end{itemize} 