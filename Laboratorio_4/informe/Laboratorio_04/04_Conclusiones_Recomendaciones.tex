\newpage
\section{CONCLUSIONES Y RECOMENDACIONES}
A lo largo del desarrollo del informe se pusieron a prueba una serie de conceptos y prácticas importantes para el manejo básico del STM32F429, así como sus GPIO, interfaces de comunicación, dispositivos para mostrar información y formas de programación. A continuación algunas conclusiones y recomendaciones recolectadas a lo largo de trabajo realizado: 

\begin{itemize}
    \item La tensión de entrada al microcontrolador es cercana a los 5 V, garantizando una operación eficiente y permitiendo la correcta evaluación de la batería, confirmando la utilidad del circuito regulador de entrada. 

    \item La pantalla del microcontrolador muestra correctamente la tensión de la batería, los ejes del acelerómetro, la temperatura y el estado de transmisión. La temperatura no es precisa debido a la limitación del sensor, pero el resto de los datos son fiables.

    \item Los LEDs indicaron correctamente los cambios de pendiente y niveles bajos de batería, proporcionando una respuesta visual inmediata y efectiva.

    \item La transmisión de datos a Thingsboard fue exitosa, y el \textit{dashboard} interactivo permite un monitoreo claro y en tiempo real de los datos del sistema, cumpliendo con los objetivos del laboratorio.

    \item Se recomienda integrar un sensor de temperatura ambiental adecuado y añadir una funcionalidad de calibración automática para los sensores del microcontrolador. Esto aumentará la precisión y confiabilidad del sistema en el monitoreo y respuesta a cambios ambientales.
\end{itemize} 