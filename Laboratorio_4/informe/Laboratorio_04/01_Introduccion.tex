\newpage
\section{INTRODUCCIÓN}
 Este trabajo aborda el desarrollo de un sistema de monitoreo de pendientes, diseñado para detectar peligros potenciales en una etapa temprana y proporcionar alertas y predicciones tempranas. El objetivo principal de este sistema es permitir a las autoridades pertinentes tomar medidas adecuadas antes de que ocurra un incidente. El sistema se construyó utilizando una placa STM32F429 Discovery kit y la biblioteca libopencm3. Para lograr el monitoreo de pendientes, se leen los ejes del giroscopio (X, Y, Z) y la temperatura. Se implementó un mecanismo de alerta mediante un LED parpadeante que se activa cuando se detecta una deformación angular mayor a 5 grados en cualquier eje. Además, se envía una notificación de advertencia al dashboard de ThingsBoard, una plataforma de IoT de código abierto basada en Java. El sistema también monitorea el nivel de la batería, cuyo rango operativo es de [0,9]V. Si el nivel de la batería se acerca al límite mínimo de operación del microcontrolador (7V), se enciende un LED de alarma parpadeante y se envía una notificación de batería baja al dashboard de ThingsBoard. Para ello, se utiliza una batería de 9V y un circuito que condiciona el nivel de voltaje dentro del rango de operación del microcontrolador. En la pantalla LCD del sistema se despliegan el nivel de batería, la temperatura, los valores de los ejes X, Y, Z y el estado de la comunicación serial/USB. Se utiliza un botón de la placa para habilitar o deshabilitar las comunicaciones por USART/USB. Además, se desarrolló un script en Python que permite leer y escribir al puerto serial/USB, enviando la información del giroscopio, temperatura y nivel de batería al dashboard de ThingsBoard.
 Para consultar el trabajo realizado, puede consultarse el siguiente repositorio de trabajo, en la rama llamada ``Lab4'': \url{https://github.com/SaboEIE/IE-0624_IS_2024_B87103.git}. El código principal en C \texttt{lab4.c}, los archivos necesarios extraídos de la librería, el archivo en python \texttt{iot.py} y el Makefile puede consultarse en la ruta \path{Laboratorio_04/src}. 
 